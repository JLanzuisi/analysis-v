\documentclass[letterpaper,12pt]{article} %Modifica el tipo de documento y el tamaño de la letra.
\usepackage[shortlabels]{enumitem}
\usepackage[spanish]{babel} 
\usepackage{mathtools}
\usepackage{hyperref}
\usepackage{wrapfig}
\usepackage[rflt]{floatflt}
\usepackage{graphicx}
\usepackage{fancyhdr} %Paquete para el header y el formato de la portada. No sugiero borrarlo!
\usepackage{float}
\usepackage{cite}
\usepackage[square,numbers]{natbib}
\usepackage{caption}
\usepackage{adjustbox}
\usepackage{parskip}
\usepackage[]{xcolor}
\usepackage{xspace}



%Fin de Préambulo


%Inicio formato de Página. Puedes establecer medidas de la página y modificar el header.

\textheight = 21cm %Medidas de la  página
\textwidth = 18cm  %Medidas de la página
\topmargin = -2cm  %Medidas de la página    
\oddsidemargin = -0.8cm %Medidas de la página
\pagestyle{fancy} %Diseño de la página


\fancyhf{}
\lhead{Departamento de Matemáticas}%%LeftHead
\chead{\includegraphics[width=1cm, height=1cm]{Imágenes/usb1.png}}%%CenterHead
%\lfoot{USB}
\rhead{Lista 1}%%RightHead

\setlength{\columnsep}{4mm}%Comandos para el formato de la página.
%\setlength{\parindent}{4em}%Sangría al comenzar un nuevo párrafo.
\setlength{\parindent}{0.5in}
%\setlength{\parindent}{4em}%Sangría al comenzar un nuevo párrafo.
\setlength{\parskip}{1em}%Distancia entre párrafos.
\renewcommand{\baselinestretch}{1.5}% Espacio entre línea y línea o interlineado.
\setlength{\headheight}{33pt}
%\fancyfoot[CE,CO]{\LaTeX{}\\\thepage} %Logo de LaTeX y pie de página.

%Fin formato de Página

%Aquí inicia el documento. Date como magnate
\begin{document}

    %LaTeX te hace el índice automáticamente conforme añades secciones en tu documento.
    \thispagestyle{empty}
           \begin{figure}[ht]
           \minipage{0.87\textwidth}
                \includegraphics[width=2cm]{Imágenes/usb1.png}
           \endminipage
           \minipage{0.32\textwidth}
                \includegraphics[height = 2.25cm ,width=2cm]{Imágenes/usb1.png}
            \endminipage
        \end{figure}
        
        \vspace{0.1cm}
        
        \begin{center}
            {\scshape\LARGE \textbf{Universidad Simón Bolívar} \par}
            {\scshape\Large Departamento de Matemáticas\par}
            
             {\LARGE Análisis V}

            % Restauramos el interlineado:
            \begin{center}
            
            {\LARGE \textit{Trimestre Enero - Marzo 2022}}

            
            {\LARGE\bfseries Lista de Ejercicios 1\par}

        {\scshape\Large Fecha de entrega: 25/02/2022\par}   

                    \LARGE  { \textbf{Profesor:}}\\%% \textbf son negritas
        \large      { MSc. Daniel Morales}
        
        \vspace{-0.5cm} 
        
        \LARGE  { \textbf{Alumno(s):}}\\%% \textbf son negritas

        \normalsize  {Jhonny Lanzuisi}
           \end{center}
        \end{center}

    \tableofcontents
    \newpage
%Inicio parte opcional. Esta parte la puedes quitar si deseas, es por si te piden formatos para
%evidencias de certificación de los laboratorios con números de cuenta o te piden abstracts en tus %trabajos.

\title{Análisis V \\\textbf{Lista de Ejercicios 1} \\ } 

\author{
  \normalsize{\texttt{Jhonny Lanzuisi}} \and
  \normalsize{\texttt{1510759}}
}
\date{\today}
\maketitle
\thispagestyle{fancy}

%Fin parte opcional

\section{Enunciados}

\begin{enumerate}
    \item Una clase de conjunto $N$ es llamada \textbf{Normal} si es cerrada bajo intersecciones enumerables de conjuntos y cerrada por uniones enumerables disjuntas de conjuntos. Demuestre que un $\sigma -$anillo es una clase Normal.
    \item Una $\sigma -$álgebra de conjuntos es definida como una clase no vacía de conjuntos que es cerrada bajo complementos y la unión enumerable de conjuntos. Muestre que una $\sigma -$álgebra es un $\sigma -$anillo que contiene a $X.$
    \item Si $\mu^{*}$ es una medida métrica exterior entonces todo conjunto abierto es $\mu^{*} -$medible.
    \item Si $E$ es un conjunto Lebesgue medible tal que, para cada $x$ en conjunto denso en casi todo punto $$\bar{\mu}\left ( E\Delta \left ( E+x \right ) \right )=0,$$ entonces $\bar{\mu}(E)=0$ ó $\bar{\mu}(E^{c})=0.$
\end{enumerate}

\section{Solución}

\newcommand{\sring}{$\sigma -$anillo\xspace}

\begin{enumerate}
    \item%
        Si $S$ es un \sring y si $E_i\in S$ para $i=1,2,\dots$
        y $E=\bigcup_{i=1}^\infty E_i$ entonces:
        \[
            \bigcap_{i=0}^\infty E_i = E - \bigcup_{i=1}^\infty (E-E_i),
        \]
        de donde se sigue que $S$ es cerrado bajo intersecciones numerables,
        pues los dos conjuntos de la diferencia, a la derecha de la igualdad,
        pertenecen a $S$.

        Como las uniones numerables de conjuntos disjuntos son un caso
        particular de las uniones numerables de conjuntos, y $S$ es cerrado
        respecto a estas últimas, se sigue que $S$ también es cerrado respecto
        a las primeras.

        Entonces, $S$ es una clase normal.

    \item%
        Sea $S$ un \sring y supongamos que $X\subset S$. Sea $E\in S$ entonces
        $X-E=E'\in S$ debido a que $S$ es cerrado bajo las diferencias.
        Además, como $S$ es un \sring se tiene que es cerrado bajo uniones numerables.

        Se sigue entonces que $S$ es una $\sigma -$álgebra.

    \item%
        Sea $U$ un conjunto abierto y $A$ un subconjunto de $X$. Sea $E=A\cap U$. Sean
        \[
            E_n = E \bigcap \left\{x\colon \rho\,(x, U')\geq\dfrac{1}{n}\right\}
        \]
        para $n=1,2,\dots$.

        Ahora $E_n \cup (A\cap U')$ son una secuencia de subconjuntos de $A$, pues cada
        $E_n$ es un subconjunto de $A$ y $A\cap U'$ también lo es. A su vez, la union
        \[
            \bigcup_{i=0}^\infty E_n \cup (A\cap U')
        \]
        es un subconjunto de $A$, puesto que la union de los $E_n$ es $A\cap U$ por construcción.

        Tenemos entonces que
        \[
            \mu^*(A)\geq\mu^*(E_n \cup (A\cap U')),
        \]
        y como $\rho\,(E_n, A\cap U')>0$, se sigue que
        \[
            \mu^*(A)\geq\mu^*(E_n)+\mu^*(A\cap U').
        \]

        Por ultimo, la construcción de los $E_n$ asegura que el $\lim_n E_n = E$
        (Ver el libro de Halmos pag 48 (8a)). Por lo que al tomar límites se obtiene
        \[
            \mu^*(A)\geq\mu^*(A\cap U)+\mu^*(A\cap U'),
        \]
        y el conjunto abirto $U$ es $\mu^*-$medible.
\end{enumerate}

\end{document}
